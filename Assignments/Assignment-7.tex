\documentclass[11pt]{article}
\usepackage{listings}
\usepackage{xcolor}
\usepackage{geometry}
\usepackage[T1]{fontenc}
\usepackage[utf8]{inputenc}
\usepackage{tcolorbox}
\usepackage{amsmath}
\usepackage{amssymb}


\usepackage{hyperref}

\geometry{a4paper, margin=1in}


\lstset{
    basicstyle=\ttfamily\footnotesize,
    keywordstyle=\color{blue}\bfseries,
    stringstyle=\color{red},
    commentstyle=\color{green!60!black},
    frame=single,
    numbers=left,
    numberstyle=\tiny\color{gray},
    showstringspaces=false,
    breaklines=true,
    captionpos=b,
    tabsize=4,
    upquote=true
}

\begin{document}

\begin{center}
    \Large{Assignment 7: Functions}
\end{center}

\section*{Task 1: Cylinder Volume Function}

\section*{Task 1.1: Cylinder Volume Function Using List}
Create a function which accepts two list inputs: \texttt{radius} and \texttt{height}, having the same length and same index for same correspoind

\begin{lstlisting}[language=Python]
def calculate_cylinder_volume(radius: list, height: list):
    # check if all elements in input list are numbers, and lists have the same length8
    # volume[i] = pi * (radius[i]**2) * height[i]

    return volume
\end{lstlisting}

\section*{Task 1.2: Cylinder Volume Function Using NumPy Array}
Please see \href{https://numpy.org/doc/stable/user/absolute_beginners.html}{NumPy Array} for the tutorial and instruction to using NumPy array.

\noindent
Try to implement using \texttt{NumPy} library, and compare \texttt{Task 1.1} and \texttt{Task 1.2} run time with \texttt{time} library.

\begin{lstlisting}[language=Python]
import time

start_time = time.time()

# program runs

end_time = time.time()

print(f"Run Time: {end_time - start_time} seconds")
\end{lstlisting}

\section*{Task 2: Derivative Function}
Create two functions, the first one accepts input \texttt{x} and return output \texttt{y}, and another function accepts two inputs: \texttt{x} and the first function to estimate slope.

\begin{lstlisting}[language=Python]
def function(x):
    y = x**2    # can be defined as others
    
    return y

def calculate_derivative(x, function):
    # calculate slope
    # m = delta_y / delta_x

    return m
\end{lstlisting}

\begin{tcolorbox}[colback=black!10!white, colframe=black!75!white, title=\textbf{Answer}]
    \vspace{3cm}
\end{tcolorbox} 


\end{document}
