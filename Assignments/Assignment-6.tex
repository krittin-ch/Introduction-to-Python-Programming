\documentclass[11pt]{article}
\usepackage{listings}
\usepackage{xcolor}
\usepackage{geometry}
\usepackage[T1]{fontenc}
\usepackage[utf8]{inputenc}
\usepackage{tcolorbox}
\usepackage{amsmath}
\usepackage{amssymb}


\usepackage{hyperref}

\geometry{a4paper, margin=1in}


\lstset{
    basicstyle=\ttfamily\footnotesize,
    keywordstyle=\color{blue}\bfseries,
    stringstyle=\color{red},
    commentstyle=\color{green!60!black},
    frame=single,
    numbers=left,
    numberstyle=\tiny\color{gray},
    showstringspaces=false,
    breaklines=true,
    captionpos=b,
    tabsize=4,
    upquote=true
}

\begin{document}

\begin{center}
    \Large{Assignment 6: Loops}
\end{center}

\section*{Task 1: Tossing a Dice}
A student wants to understand the probability of obtaining a sum of 4 or 10 when tossing two six-sided dice by using a sufficiently large sample size (1,000 trials).\\

\noindent
According to the student calculation:
\[
\begin{aligned}
    P[\text{sum } = 4 \text{ or } 10] &= P[x = \{1, 3\}] + P[x = \{2, 2\}] + P[x = \{4, 6\}] + P[x = \{5, 5\}]\\
    &= \frac{2}{36} + \frac{1}{36} + \frac{2}{36} + \frac{1}{36}\\
    \therefore \hspace{0.1cm} P[\text{sum } = 4 \text{ or } 10] &= \frac{1}{6}
\end{aligned}
\]

\noindent
Construct a Python program to check the student's hypothesis. (Hint: using \texttt{random} library.)

\begin{tcolorbox}[colback=black!10!white, colframe=black!75!white, title=\textbf{Answer}]
    \vspace{3cm}
\end{tcolorbox}    


\section*{Task 2: Taylor's Series}

By using Taylor's series, A function can be expressed in series of the function derivatives.\\
\noindent
The following equations are example of Taylor's series:
\[
e^x = \sum_{n=0}^\infty \frac{x^n}{n!} \, , \quad \sin(x) = \sum_{n=0}^\infty \frac{(-1)^n x^{2n + 1}}{(2n + 1)!} \, , \quad \cos(x) = \sum_{n=0}^\infty \frac{(-1)^n x^{2n}}{(2n)!}\\
\]

\noindent
Generally, the series converges as \(n\) approaches infinity. However, for this assignment, please approximate \(\sin(x)\) and \(\cos(x)\) when \(x=30^{\circ} \text{ and } n=10\), and check their trigonometric identity\\(\(\sin^2(x) + \cos^2(x) = 1\).)

\begin{tcolorbox}[colback=black!10!white, colframe=black!75!white, title=\textbf{Answer}]
    \vspace{3cm}
\end{tcolorbox} 


\end{document}
