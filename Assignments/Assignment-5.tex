\documentclass[11pt]{article}
\usepackage{listings}
\usepackage{xcolor}
\usepackage{geometry}
\usepackage[T1]{fontenc}
\usepackage[utf8]{inputenc}
\usepackage{tcolorbox}

\usepackage{hyperref}

\geometry{a4paper, margin=1in}


\lstset{
    basicstyle=\ttfamily\footnotesize,
    keywordstyle=\color{blue}\bfseries,
    stringstyle=\color{red},
    commentstyle=\color{green!60!black},
    frame=single,
    numbers=left,
    numberstyle=\tiny\color{gray},
    showstringspaces=false,
    breaklines=true,
    captionpos=b,
    tabsize=4,
    upquote=true
}

\begin{document}

\begin{center}
    \Large{Assignment 5: Decisions}
\end{center}

\section*{Task 1: Grading System}
A professor needs to grade his students using Python if-elif-else statements.
Given the grading criteria: A (> 80) B (70 - 80) C (60 - 70) D (50 - 60) F (< 50).\\
If a student receives an F, but their assignment score is greater than 25, the student is given a chance to take a retest.\\

\noindent
Construct a Python program that accepts two integer inputs: test score (not more than 60) and assignment score (not more than 40). The program should print the student's grade, total score, and indicate if the student needs a retest.



\begin{tcolorbox}[colback=black!10!white, colframe=black!75!white, title=\textbf{Answer}]
    \vspace{3cm}
\end{tcolorbox}    


\section*{Task 2: Pythagoras Theorem}

The program accepts three inputs: \texttt{a}, \texttt{b}, and \texttt{c}, 
which represent the sides of a triangle. 
It checks whether these numbers can form a valid triangle. 
If they can, the program further checks if the triangle satisfies 
the Pythagorean theorem, indicating that it is a right triangle.

\begin{tcolorbox}[colback=black!10!white, colframe=black!75!white, title=\textbf{Answer}]
    \vspace{3cm}
\end{tcolorbox} 


\end{document}
