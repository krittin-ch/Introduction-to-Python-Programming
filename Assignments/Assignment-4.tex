\documentclass[11pt]{article}
\usepackage{listings}
\usepackage{xcolor}
\usepackage{geometry}
\usepackage[T1]{fontenc}
\usepackage[utf8]{inputenc}
\usepackage{tcolorbox}

\usepackage{hyperref}

\geometry{a4paper, margin=1in}


\lstset{
    basicstyle=\ttfamily\footnotesize,
    keywordstyle=\color{blue}\bfseries,
    stringstyle=\color{red},
    commentstyle=\color{green!60!black},
    frame=single,
    numbers=left,
    numberstyle=\tiny\color{gray},
    showstringspaces=false,
    breaklines=true,
    captionpos=b,
    tabsize=4,
    upquote=true
}

\begin{document}

\begin{center}
    \Large{Assignment 4: Objects}
\end{center}

\section*{Questions}

\begin{enumerate}
    \item {
        What are \texttt{id()} and \texttt{type()}?
        \begin{tcolorbox}[colback=black!10!white, colframe=black!75!white, title=\textbf{Answer}]
            \vspace{2.5cm}
        \end{tcolorbox}
    }
    \item {
        Given the code \verb|print("Hello,\tI\'m Mana.\nI want to be a \"demon slayer\".")|\\
        What does it print?

        \begin{tcolorbox}[colback=black!10!white, colframe=black!75!white, title=\textbf{Answer}]
            \vspace{2.5cm}
        \end{tcolorbox}
    }
    \item {
        Consider the following program:

\begin{lstlisting}[language=Python]
s = chr(85) + 2*chr(73) + chr(65) + chr(32)
n = ord("~") - ord("x")

new_s = n * s
print(f"The new sentence is {new_s}")
\end{lstlisting}
\noindent
    What is its output?
    \begin{tcolorbox}[colback=black!10!white, colframe=black!75!white, title=\textbf{Answer}]
        \vspace{2.5cm}
    \end{tcolorbox}
    }
    \item {
        Given \verb|s = "Minecraft"|, what is the value and type of \verb|ord(s[1])|?
        \begin{tcolorbox}[colback=black!10!white, colframe=black!75!white, title=\textbf{Answer}]
            \vspace{2.5cm}
        \end{tcolorbox}    
    }
\end{enumerate}

\section*{Task 1: Formating Output}
Professor John want to estimate the Euler's number using limit definition:

\[ e = \lim_{n \rightarrow \infty} \left(1 + \frac{1}{n} \right) ^ n \]

\noindent
As \(n\) approches infinity, he assigns \(n = 5 \times 10^3\) and wants to print its value formatted with 3 total digits 
before the decimal point and 5 digits after the decimal point, padded accordingly. What is the value that he print?

\begin{tcolorbox}[colback=black!10!white, colframe=black!75!white, title=\textbf{Answer}]
    \vspace{3cm}
\end{tcolorbox}    


\section*{Task 2: Lists and Tuples}
\subsection*{Task 2.1: Lists}
Consider the following program:

\begin{lstlisting}[language=Python]
num_list = [1, 2, "Tanjiro", 4, 5]

x = num_list[0] + num_list[-1]
y = num_list[1] + num_list[-2]
num_list[2] = x - y

print(f"x - y = {num_list[2]}")
\end{lstlisting}

\noindent
What is the program output?
\begin{tcolorbox}[colback=black!10!white, colframe=black!75!white, title=\textbf{Answer}]
    \vspace{3cm}
\end{tcolorbox}    

\clearpage

\subsection*{Task 2.2: Tuples}
Consider the following program:

\begin{lstlisting}[language=Python]
num_list = (1, 2, "Nezuko", 4, 5)

x = num_list[0] + num_list[-1]
y = num_list[1] + num_list[-2]
num_list[2] = x - y

print(f"x - y = {num_list[2]}")
\end{lstlisting}

\noindent
What is the program output? Discuss the result.
\begin{tcolorbox}[colback=black!10!white, colframe=black!75!white, title=\textbf{Answer}]
    \vspace{3cm}
\end{tcolorbox} 

\subsection*{Task 2.3: List and Tuple Applications}
Why are tuples important, even though they are immutable? Additionally, when should we use a tuple instead of a list?
\begin{tcolorbox}[colback=black!10!white, colframe=black!75!white, title=\textbf{Answer}]
    \vspace{3cm}
\end{tcolorbox} 

\end{document}
