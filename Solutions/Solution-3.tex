\documentclass[11pt]{article}
\usepackage{listings}
\usepackage{xcolor}
\usepackage{geometry}
\usepackage[T1]{fontenc}
\usepackage[utf8]{inputenc}
\usepackage{tcolorbox}
\usepackage{amsmath}

\usepackage{hyperref}

\geometry{a4paper, margin=1in}


\lstset{
    basicstyle=\ttfamily\footnotesize,
    keywordstyle=\color{blue}\bfseries,
    stringstyle=\color{red},
    commentstyle=\color{green!60!black},
    frame=single,
    numbers=left,
    numberstyle=\tiny\color{gray},
    showstringspaces=false,
    breaklines=true,
    captionpos=b,
    tabsize=4,
    upquote=true
}

\begin{document}

\begin{center}
    \Large{Solution 3: Expressions}
\end{center}

\section*{Questions}

\begin{enumerate}
    \item {
        What are implicit and explicit type conversion? Also, give an example for each of them.
        \begin{tcolorbox}[colback=black!10!white, colframe=black!75!white, title=\textbf{Answer}]
            
            \textbf{Implicit type conversion} happens automatically through mathematical operation, without manual intervention.
            For example, 

    \begin{lstlisting}[language=Python]
    x = 9
    y = 3
    z = x/y     # z = 3.0 
    \end{lstlisting}    
    
        \textbf{Explicit type conversion} requires manual work in order to reassign data type.
            For example, 

    \begin{lstlisting}[language=Python]
    x = 10
    z = float(x) # z = 10.0
    \end{lstlisting}    
            
        \end{tcolorbox}
    }
    \item {
        What are 2 types of floating-point error, and how to avoid them?
        \begin{tcolorbox}[colback=black!10!white, colframe=black!75!white, title=\textbf{Answer}]
            There are \textbf{round-off error} and \textbf{overflow error}.\\
            \par
            \textbf{Round-off error} happens from attempting to make a number too precise. Round-off error can be ameliorated by using function \texttt{round()} to get desired error.\\
            \textbf{Overflow error} exists when a number is too large, and can be eliminated by using different data type, or other libraries.\\

            \par
            See this \href{https://www.geeksforgeeks.org/overflowerror-convert-int-large-to-float-in-python/}{link} for other solutions.
        \end{tcolorbox}
    }
    \item {
        How can programming be adpated in science and business?  
        \begin{tcolorbox}[colback=black!10!white, colframe=black!75!white, title=\textbf{Answer}]
            Programming can be applied in various scenarios. 
            For example, in the business sector, it is used for tasks such as monthly sales prediction, fraud detection, chatbots, websites, and data center management. 
            In science, programming can be applied to computational chemistry (\href{https://youtu.be/7q8Uw3rmXyE?si=auYcRjVZkxvq_BN5}{AlphaFold}), physics simulation (\href{https://youtu.be/kRlhlCWplqk?si=skcE9BLBjUz8EkEO}{NASA}).
            Additionally, it is widely used in game development such as \href{https://youtu.be/C8YtdC8mxTU?si=wq3iSpSbeevr7HIx}{game graphics}.
        \end{tcolorbox}
    }
\end{enumerate}

\clearpage

\section*{Task 1: Rounding Numbers}
Given \texttt{x = 5/9}, round the variable \texttt{x} to 2 decimal digits.

\begin{lstlisting}[language=Python]
x = 5/9
x = round(x, 2) # x = 0.56 
\end{lstlisting}    

How would the rounded \texttt{x} affect the further computation?
\begin{tcolorbox}[colback=black!10!white, colframe=black!75!white, title=\textbf{Answer}]
    Since the original value is \texttt{x = 5/9}, the error contribution caused by rounding to two decimal places can be calculated as follows:
    \[
        \begin{aligned}
            x &= \frac{5}{9} \ \text{ and } \ x' = 0.56 \\
            \text{error} &= \left| \frac{x - x'}{x} \right| \times 100\%, \\
            \text{error} &= 0.8 \%
        \end{aligned}
    \]
\end{tcolorbox}
\section*{Task 2: Unit Conversion}
Convert 12,345 seconds into hours, minutes, and seconds, and print it in the \texttt{hh:mm:ss} format. Also, answer how many hours in \texttt{hhh.hhhhh} format (for example, 001.12300 hours).
\begin{lstlisting}[language=Python]
time = 12345

s = time
m = s // 60
h = m // 60

m = m - h * 60
s = s - (h * 3600 + m * 60)
print(f"{h:02d}:{m:02d}:{s:02d}")

h = time / 3600
print(f"{h:03.5f} hours")

\end{lstlisting}    

\end{document}
