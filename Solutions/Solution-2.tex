\documentclass[11pt]{article}
\usepackage{listings}
\usepackage{xcolor}
\usepackage{geometry}
\usepackage[T1]{fontenc}
\usepackage[utf8]{inputenc}
\usepackage{tcolorbox}

\usepackage{hyperref}

\geometry{a4paper, margin=1in}


\lstset{
    basicstyle=\ttfamily\footnotesize,
    keywordstyle=\color{blue}\bfseries,
    stringstyle=\color{red},
    commentstyle=\color{green!60!black},
    frame=single,
    numbers=left,
    numberstyle=\tiny\color{gray},
    showstringspaces=false,
    breaklines=true,
    captionpos=b,
    tabsize=4,
    upquote=true
}

\begin{document}

\begin{center}
    \Large{Solution 2: Simple Application of Statements}
\end{center}

\section*{Task 1: BMI Calculator}
Write a Python program that asks the user for their name, height, and weight, then calculate and print the BMI value with the user's name.

\begin{lstlisting}[language=Python]
name = input("Enter your name: ")
height = input("Enter your height: ")
weight = input("Enter your weight: ")

# BMI Calculation
BMI = weight / height**2 # weight (kg) and height (m)

print(f"Hello {name}! Your BMI is {BMI}.")
\end{lstlisting}

\section*{*Task 2: Combination Calculation (\(\mathbf{C_r^n = \frac{n!}{(n-r)!r!}}\))}
Define \(C_r^n\) as number of ways to choose r items from n items without repetition and without order.
The task is to assign two inputs: \(n\) and \(r\), and use math library (see url attached) to compute the associated combination.
url: \url{https://docs.python.org/3/library/math.html#}

\begin{lstlisting}[language=Python]
# Import library 
import math

n = input("n = ")
r = input("r = ")

num_c = math.comb(n, r)

"""
Another method

num_c = math.factorial(n) / (math.factorial(n-r) * math.factorial(r))

"""

print(f"The combination of choosing {r} items from {n} items is {num_c}")
\end{lstlisting}

\newpage

\section*{*Task 3: Acid-Base Calculation}
2 L of solution of \(5 \times 10^-3\) M HCL and \(3 \times 10^-3\) M AgNO\(_3\) mixing together. Calculate the followings:
\begin{enumerate}
    \item pH of the solution
    \item Mass of precipitation
    \item Mass of pure water (without HCL and AgNO\(_3\)) 
\end{enumerate}

\begin{lstlisting}[language=Python]
# Import library 
import math

HCL_M = 5e-3    # equals to 5 * 10**-3
AgNO3_M = 3e-3  # equals to 3 * 10**-3
V = 2           # 2 L of water

total_H = HCL_M + AgNO3_M
precipitation = AgNO3_M # the precipitation is AgNO3

pH = -math.log10(total_H)

print(f"The pH of the solution is {pH}")    # 2.097

AgCl_molar_mass = 143.32   # g/mol

AgCl_mass = precipitation * V * AgCl_molar_mass

print(f"The precipitation is AgCl, and it weighs {AgCl_mass}")  # 0.860 g

HCl_density = 1.18      # g/cm3
HCL_molar_mass = 36.46  # g/mol

AgNO3_density = 4.35        # g/cm3
AgNO3_molar_mass = 169.87   # g/mol

HCl_mass = HCL_M * V * HCL_molar_mass
AgNO3_mass = AgNO3_M * V * AgNO3_molar_mass

water_density = 0.997 # g/cm3

water_mass = V * (water_density * 1000) - (HCl_mass + AgNO3_mass)

print(f"Pure water mass is {water_mass}")   # 1992.616 g
    
\end{lstlisting}


\end{document}
