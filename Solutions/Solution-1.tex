\documentclass[11pt]{article}
\usepackage{listings}
\usepackage{xcolor}
\usepackage{geometry}
\usepackage[T1]{fontenc}
\usepackage[utf8]{inputenc}

\geometry{a4paper, margin=1in}


\lstset{
    basicstyle=\ttfamily\footnotesize,
    keywordstyle=\color{blue}\bfseries,
    stringstyle=\color{red},
    commentstyle=\color{green!60!black},
    frame=single,
    numbers=left,
    numberstyle=\tiny\color{gray},
    showstringspaces=false,
    breaklines=true,
    captionpos=b,
    tabsize=4,
    upquote=true
}

\begin{document}

\begin{center}
    \Large{Solution 1: Statements}
\end{center}


\section*{Part 2: Variables and Naming Rules}

\subsection*{Task 2}
Identify the errors in the following variable assignments and correct them:

\begin{lstlisting}[language=Python]
# Incorrect code:
1name = "Alice"       # Variable name starts with a number.
def = "Reserved"      # Using a reserved keyword.
user-name = "John"    # Hyphens are not allowed in variable names.

# Corrected code:
name1 = "Alice"             # Variable name starts with a letter.
reserved_word = "Reserved"  # Avoid reserved keywords.
user_name = "John"          # Use underscores instead of hyphens.
\end{lstlisting}

\section*{Part 3: String Basics}

\subsection*{Task 3}
Write a Python program to demonstrate the use of quotation marks, string length, and concatenation.

\begin{lstlisting}[language=Python]
# Quotation marks:
single_quote = 'This is a string.'
double_quote = "This is also a string."

# String length:
message = "Hello, Python!"
print(len(message))  # Output: 13

# String concatenation:
first_name = "Alice"
last_name = "Johnson"
full_name = first_name + " " + last_name
print(full_name)  # Output: Alice Johnson
\end{lstlisting}

\newpage

\section*{Part 4: Number Basics}

\subsection*{Task 4}
Write a Python program to demonstrate data types, basic arithmetic operations, and order of precedence.

\begin{lstlisting}[language=Python]
# Data types:
integer_number = 10
float_number = 5.5
print(type(integer_number))  # Output: <class 'int'>
print(type(float_number))    # Output: <class 'float'>

# Basic arithmetic operations:
addition = 5 + 3
subtraction = 10 - 4
multiplication = 7 * 6
division = 15 / 3

print(addition, subtraction, multiplication, division)  # 8 6 42 5.0

# Order of precedence:
# Exponentiation -> Multiplication -> Addition -> Subtraction
result = 5 + 2 * 3 ** 2 - 1  
print(result)  # Output: 22
\end{lstlisting}

\section*{Part 5: Error Handling}

\subsection*{Task 5}
Explain what an error is, identify which line of code contains an error, and provide a solution.

\subsubsection*{Task 5.1: Syntax Error}
\begin{lstlisting}[language=Python]
# Incorrect code:
print("Hello World!"  # Missing closing parenthesis

# Explanation:
# A SyntaxError occurs because the closing parenthesis is missing on line 2.

# Corrected code:
print("Hello World!")
\end{lstlisting}

\subsubsection*{Task 5.2: NameError}
\begin{lstlisting}[language=Python]
# Incorrect code:
print(variable)  # 'variable' is not defined
variable = 5

# Explanation:
# A NameError occurs because 'variable' has not been defined before use on line 2.

# Corrected code:
variable = 5
print(variable)
\end{lstlisting}

\subsubsection*{Task 5.3: ZeroDivisionError}
\begin{lstlisting}[language=Python]
# Incorrect code:
result = 10 / 0  # Division by zero is not allowed

# Explanation:
# A ZeroDivisionError occurs because a number divided by zero on line 2.

# Corrected code:
result = 10 / 2  # Use a non-zero divisor
print(result)
\end{lstlisting}

\subsubsection*{Task 5.4: TypeError}
\begin{lstlisting}[language=Python]
# Incorrect code:
# input() always return a string
# int() is used to convert to 'int'
x = int(input("Please enter a number from 1-10: ")) 
print("Your input is ", x)          # x = 3

y = input("Please enter your second number from 1-10: ")
print("Your second number is ", y)  # y = 5
print()

print("The sum of two numbers is x" + y)
# The sum of two numbers is x 5
print("The product of two numbers is ", x * y)
"""
Traceback (most recent call last):
  File "c:\Users\kritt\Desktop\mana\temp_1.py", line 10, in <module>
    print("The product of two numbers is ", x * y)
                                            ~~^~~
TypeError: can't multiply sequence by non-int of type 'str
"""
\end{lstlisting}

\begin{lstlisting}[language=Python]
# Corrected code:
# input() always return a string
# int() is used to convert to 'int'
x = int(input("Please enter a number from 1-10: ")) 
print("Your input is ", x)          # x = 3

y = int(input("Please enter your second number from 1-10: ")) # Add int()
print("Your second number is ", y)  # y = 5
print()

# remove 'x' from the string, and add with 'y'
print("The sum of two numbers is ", x + y) 
# The sum of two numbers is 8
print("The product of two numbers is ", x * y) 
# The product of two numbers is 15
\end{lstlisting}


\end{document}
