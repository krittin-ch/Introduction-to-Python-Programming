\documentclass[11pt]{article}
\usepackage{listings}
\usepackage{xcolor}
\usepackage{geometry}
\usepackage[T1]{fontenc}
\usepackage[utf8]{inputenc}
\usepackage{tcolorbox}
\usepackage{upquote}

\usepackage{hyperref}

\geometry{a4paper, margin=1in}


\lstset{
    basicstyle=\ttfamily\footnotesize,
    keywordstyle=\color{blue}\bfseries,
    stringstyle=\color{red},
    commentstyle=\color{green!60!black},
    frame=single,
    numbers=left,
    numberstyle=\tiny\color{gray},
    showstringspaces=false,
    breaklines=true,
    captionpos=b,
    tabsize=4,
    upquote=true
}

\begin{document}

\begin{center}
    \Large{Solution 4: Objects}
\end{center}

\section*{Questions}

\begin{enumerate}
    \item {
        What are \texttt{id()} and \texttt{type()}?
        \begin{tcolorbox}[colback=black!10!white, colframe=black!75!white, title=\textbf{Answer}]
            \texttt{id()} and \texttt{type()} are a built-in function. While \texttt{id()} returns an unqiue integer, representing the memory address of the object, \texttt{type()} returns the class which the object belongs to.
        \end{tcolorbox}
    }
    \item {
        Given the code \verb|print("Hello,\tI'm Mana.\nI want to be a \"demon slayer\".")|\\
        What does it print?

\begin{lstlisting}[language=Python]
print("Hello,\tI\'m Mana.\nI want to be a \"demon slayer\".")

# Hello,  I'm Mana.
# I want to be a "demon slayer".
\end{lstlisting}
                
    }
    \item {
        Consider the following program:

\begin{lstlisting}[language=Python]
s = chr(85) + 2*chr(73) + chr(65) + chr(32)
n = ord("~") - ord("x")

new_s = n * s
print(f"The new sentence is {new_s}")
\end{lstlisting}
    \noindent
    What is its output?
    \begin{tcolorbox}[colback=black!10!white, colframe=black!75!white, title=\textbf{Answer}]
        The program prints: \verb|The new sentence is UIIA UIIA UIIA UIIA UIIA UIIA|
    \end{tcolorbox}
    }
    \item {
        Given \verb|s = "Minecraft"|, what is the value and type of \verb|ord(s[1])|?
        \begin{tcolorbox}[colback=black!10!white, colframe=black!75!white, title=\textbf{Answer}]
            The value of \verb|ord(s[1])| is \verb|'i'|, and its type is \verb|<class 'str'>|, which is of type string.
        \end{tcolorbox}    
    }
\end{enumerate}

\section*{Task 1: Formating Output}
Professor John want to estimate the Euler's number using limit definition:

\[ e = \lim_{n \rightarrow \infty} \left(1 + \frac{1}{n} \right) ^ n \]

\noindent
As \(n\) approches infinity, he assigns \(n = 5 \times 10^3\) and wants to print its value formatted with 3 total digits 
before the decimal point and 5 digits after the decimal point, padded accordingly. What is the value that he print?

\begin{lstlisting}[language=Python]
n = 5e3 # n = 5 x 10^3
e = (1 + 1/n) ** n

print(f'The Euler\'s Number is {e:09.5f} when n = {int(n):,d}')
# The Euler's Number is 02.71801 when n = 5,000
\end{lstlisting}

\section*{Task 2: Lists and Tuples}
\subsection*{Task 2.1: Lists}
Consider the following program:

\begin{lstlisting}[language=Python]
num_list = [1, 2, "Tanjiro", 4, 5]

x = num_list[0] + num_list[-1]
y = num_list[1] + num_list[-2]
num_list[2] = x - y

print(f"x - y = {num_list[2]}")
\end{lstlisting}

\noindent
What is the program output?
\begin{tcolorbox}[colback=black!10!white, colframe=black!75!white, title=\textbf{Answer}]
    The program shows \verb|x - y = 0|, which is the new value of the second index of \verb|num_list|.
\end{tcolorbox}    

\clearpage

\subsection*{Task 2.2: Tuples}
Consider the following program:

\begin{lstlisting}[language=Python]
num_list = (1, 2, "Nezuko", 4, 5)

x = num_list[0] + num_list[-1]
y = num_list[1] + num_list[-2]
num_list[2] = x - y

print(f"x - y = {num_list[2]}")
\end{lstlisting}

\noindent
What is the program output? Discuss the result.
\begin{tcolorbox}[colback=black!10!white, colframe=black!75!white, title=\textbf{Answer}]
    The program could not execute, and raised the following error:\\
    \verb|TypeError: 'tuple' object does not support item assignment|\\
    This error represent the immutable property of the tuple data type.
\end{tcolorbox} 

\subsection*{Task 2.3: List and Tuple Applications}
Why are tuples important, even though they are immutable? Additionally, when should we use a tuple instead of a list?
\begin{tcolorbox}[colback=black!10!white, colframe=black!75!white, title=\textbf{Answer}]
    Tuple is an \textbf{ordered}, \textbf{immutable} collection of elements, unlike list which is \textbf{mutable}.  
    This property offers \textbf{data integrity} ensuring data consistency which can be applied in banking industries.
    Additionally, due to the mutability of the list data type, its memory alocation is dynamic meaning that it is changable. This may cause extra \textbf{time complexity}, which may, in turn, make the system slower.
\end{tcolorbox} 

\end{document}
