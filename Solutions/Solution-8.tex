\documentclass[11pt]{article}
\usepackage{listings}
\usepackage{xcolor}
\usepackage{geometry}
\usepackage[T1]{fontenc}
\usepackage[utf8]{inputenc}
\usepackage{tcolorbox}
\usepackage{amsmath}
\usepackage{amssymb}


\usepackage{hyperref}

\geometry{a4paper, margin=1in}


\lstset{
    basicstyle=\ttfamily\footnotesize,
    keywordstyle=\color{blue}\bfseries,
    stringstyle=\color{red},
    commentstyle=\color{green!60!black},
    frame=single,
    numbers=left,
    numberstyle=\tiny\color{gray},
    showstringspaces=false,
    breaklines=true,
    captionpos=b,
    tabsize=4,
    upquote=true
}

\begin{document}

\begin{center}
    \Large{Solution 8: Revision from Chapter 1 to 6}
\end{center}

\section*{Task 1: List Set Operation}
Implement union and intersection operation by creating a function, using list data structure.

\begin{lstlisting}[language=Python]
def union_list(list_1, list_2):
    # list_1 union list_2

    union_list = list_2.copy()
    for i in list_1:
        if i in list_2:
            continue
            union_list.append(i)
    
    return union_list

def intersection_list(list_1, list_2):
    # list_1 intersect list_2

    intersect_list = []

    for i in list_1:
        if i in list_2:
            intersect_list.append(i)
    
    return intersect_list
\end{lstlisting}


\section*{Task 2: Masking Elements}
Create a function that accepts a list of integer input and creates a new list that mark \texttt{True} if that element is less than 15 but more than 5 and \texttt{False} if not.

\begin{lstlisting}[language=Python]
def masking_list(ls):
    """
    if ...:
        mask[i] = True
    else:
        mask[i] = False
    """

    mask = [i > 5 and i < 15 for i in ls]

    return mask
\end{lstlisting}

\section*{Task 3: Arc Length}
Define arc length \(s\):
\[
\begin{aligned}
    s &= \sqrt{(dx)^2 + (dy)^2}\\
    &= \sqrt{(dx)^2 \cdot \left(1 + \left(\frac{dy}{dx} \right)^2 \right)}\\
    \therefore \hspace{0.3cm} s &= \sqrt{1 + \left(\frac{dy}{dx} \right)^2} \cdot dx
\end{aligned}
\]

We can conclude that small arc length \(s\) is the sum of  \(\sqrt{1 + \text{slope}^2}\) times \(\Delta x\).

That is,
\[
\begin{aligned}
    S &= \int_a^b \sqrt{1 + \left( \frac{dy}{dx} \right)^2} dx \\
    &= \sum_{x=a}^b \sqrt{1 + \left( \frac{\Delta y}{ \Delta x} \right)^2} \cdot \Delta x \; ; \text{ for very small step size } h
\end{aligned}
\]

\begin{lstlisting}[language=Python]
import math

def func(x):
    y = x**2

    return y

def find_slope(x, h, func):
    delta_y = func(x + h) - func(x)
    delta_x = h

    m = delta_y / delta_x

    return m, h

def delta_arc_length(x, h, func):
    slope = find_slope(x, h, func)

    delta_s = math.sqrt(1 + slope**2) * h

    return delta_s

def arc_length(a, b, func):
    h = 0.01

    S = 0
    for i in range(a, b, h):
        S += delta_arc_length(i, h, func)

    return S
    
\end{lstlisting}

\end{document}
