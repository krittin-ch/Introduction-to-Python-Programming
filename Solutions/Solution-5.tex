\documentclass[11pt]{article}
\usepackage{listings}
\usepackage{xcolor}
\usepackage{geometry}
\usepackage[T1]{fontenc}
\usepackage[utf8]{inputenc}
\usepackage{tcolorbox}

\usepackage{hyperref}

\geometry{a4paper, margin=1in}


\lstset{
    basicstyle=\ttfamily\footnotesize,
    keywordstyle=\color{blue}\bfseries,
    stringstyle=\color{red},
    commentstyle=\color{green!60!black},
    frame=single,
    numbers=left,
    numberstyle=\tiny\color{gray},
    showstringspaces=false,
    breaklines=true,
    captionpos=b,
    tabsize=4,
    upquote=true
}

\begin{document}

\begin{center}
    \Large{Solution 5: Decisions}
\end{center}

\section*{Task 1: Grading System}
A professor needs to grade his students using Python if-elif-else statements.
Given the grading criteria: A (> 80) B (70 - 80) C (60 - 70) D (50 - 60) F (< 50).\\
If a student receives an F, but their assignment score is greater than 25, the student is given a chance to take a retest.\\

\noindent
Construct a Python program that accepts two integer inputs: test score (not more than 60) and assignment score (not more than 40). The program should print the student's grade, total score, and indicate if the student needs a retest.

\begin{lstlisting}[language=Python]
test_score = int(input("Test Score: "))
assignment_score = int(input("Assignment Score: "))

if test_score > 60: 
    print(f"Test score is {test_score}. It should not more than 60")
elif assignment_score > 40: 
    print(f"Assignment score is {assignment_score}. It should not more than 40")
else:
    pass

total_score = test_score + assignment_score

if_retest = False

if total_score > 80:
    grade = "A"
elif 80 >= total_score > 70:
    grade = "B"
elif 70 >= total_score > 60:
    grade = "C"
elif 60 >= total_score > 50:
    grade = "D"
else:
    grade = "F"
    if assignment_score > 25:
        if_retest = True
    else:
        pass

print(f"The student's grade is {grade} \
      with a total score of {total_score}. \
      Retest required: {if_retest}.")
\end{lstlisting}

\noindent
Note that the program continues to run until the end, as it is not designed to handle errors.

\clearpage
\section*{Task 2: Pythagoras Theorem}

The program accepts three inputs: \texttt{a}, \texttt{b}, and \texttt{c}, 
which represent the sides of a triangle. 
It checks whether these numbers can form a valid triangle. 
If they can, the program further checks if the triangle satisfies 
the Pythagorean theorem, indicating that it is a right triangle.\\

\begin{lstlisting}[language=Python]
a = float(input("Side 1: "))
b = float(input("Side 2: "))
c = float(input("Side 3: "))

# Find largest side
max_length = max(a, b, c)

if max_length == a:
    s_1 = b 
    s_2 = c
elif max_length == b:
    s_1 = a
    s_2 = c
else:
    s_1 = a
    s_2 = b

if not s_1 + s_2 > max_length:
    print(f"Impossible to construct a triangle.")
else:
    print(f"Possible to construct a triangle.")

if max_length**2 == s_1**2 + s_2**2:
    print("The triangle is a right-angled triangle.")
else:
    print("The triangle is not a right-angled triangle.")

\end{lstlisting}

\end{document}
