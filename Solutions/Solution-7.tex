\documentclass[11pt]{article}
\usepackage{listings}
\usepackage{xcolor}
\usepackage{geometry}
\usepackage[T1]{fontenc}
\usepackage[utf8]{inputenc}
\usepackage{tcolorbox}
\usepackage{amsmath}
\usepackage{amssymb}


\usepackage{hyperref}

\geometry{a4paper, margin=1in}


\lstset{
    basicstyle=\ttfamily\footnotesize,
    keywordstyle=\color{blue}\bfseries,
    stringstyle=\color{red},
    commentstyle=\color{green!60!black},
    frame=single,
    numbers=left,
    numberstyle=\tiny\color{gray},
    showstringspaces=false,
    breaklines=true,
    captionpos=b,
    tabsize=4,
    upquote=true
}

\begin{document}

\begin{center}
    \Large{Solution 7: Functions}
\end{center}

\section*{Task 1: Cylinder Volume Function}

\section*{Task 1.1: Cylinder Volume Function Using List}
Create a function which accepts two list inputs: \texttt{radius} and \texttt{height}, having the same length and same index for same correspoind

\begin{lstlisting}[language=Python]
import math

def calculate_cylinder_volume(radius: list, height: list):
    # check if all elements in input list are numbers, and lists have the same length
    n = len(radius)

    assert n == len(height), f"radius contains {n}, while height contains {height}. Dimension mismatch."
    
    volume = []

    for i in range(n):
        if isintance(radius[i], (int, float)) and isintance(height[i], (int, float)):
            v.append(math.pi * radius[i]**2 * height[i])

    return volume
\end{lstlisting}

\section*{Task 1.2: Cylinder Volume Function Using NumPy Array}
Please see \href{https://numpy.org/doc/stable/user/absolute_beginners.html}{NumPy Array} for the tutorial and instruction to using NumPy array.

\begin{lstlisting}[language=Python]
    import numpy as np
    
    def calculate_cylinder_volume(radius: nd.array, height: nd.array):
        # check if all elements in input list are numbers, and lists have the same length
        
        assert radius.shape == height.shape, "Dimension mismatch."

        assert (np.issubdtype(radius.dtype, np.number) and np.issubdtype(height.dtype, np.number)), "Some elements are not numbers"

        volume = np.pi * np.pow(radius, 2) * height
        return volume
    \end{lstlisting}
    
\noindent

\section*{Task 2: Derivative Function}
Create two functions, the first one accepts input \texttt{x} and return output \texttt{y}, and another function accepts two inputs: \texttt{x} and the first function to estimate slope.

\begin{lstlisting}[language=Python]
def func(x):
    y = x**2    # can be defined as others
    
    return y

def calculate_derivative(x, func):
    # calculate slope
    # m = delta_y / delta_x

    h = 0.01 # step size
    delta_y = func(x + h) - func(x)
    delta_x = h

    m = delta_y / delta_x

    return m
\end{lstlisting}

\begin{tcolorbox}[colback=black!10!white, colframe=black!75!white, title=\textbf{Answer}]
    \vspace{3cm}
\end{tcolorbox} 

\section*{Task 3: Integration Function}
Create two functions, the first one accepts input \texttt{x} and return output \texttt{y}, and another function accepts two inputs: \texttt{x} and the first function to area under curve.

\begin{lstlisting}[language=Python]
def func(x):
    y = x**2    # can be defined as others
    
    return y

def calculate_integration(func, a, b):
    # calculate slope
    # delta_s = func(x) * delta_x

    s = 0
    h = 0.01 # step size

    for i in range(a, b, h):
        s += func(i) * h
    
    return S
\end{lstlisting}

\begin{tcolorbox}[colback=black!10!white, colframe=black!75!white, title=\textbf{Answer}]
    \vspace{3cm}
\end{tcolorbox} 

\end{document}
